\documentclass[11pt]{article}
\usepackage{amsmath}
\usepackage{array}
\usepackage{multirow}
\newcolumntype{W}{>{\centering\arraybackslash} m{1cm} }
\title{My template}
\author{Insert Your name}
\date{\today}
%end preamble
\begin{document}
\maketitle
\section{Graphics}
\section{The linear regression model}
\subsection{In line equations}
\label{sect.myfirst}
Writing symbols such as $Y$, $X$ or $\boldsymbol{\beta}$ is easy. I can
even create a whole in-line formula, such as $Y=X\beta+\epsilon$, or even better
$\textbf{Y}=X\boldsymbol{\beta}+\boldsymbol{\epsilon}$. To create spaces,
I can use the $\sim$ symbol: for instance $~~X\boldsymbol\beta~~$ has more
space than $X\boldsymbol\beta$.
\subsection{Full line equations}
\subsubsection{Numbered equations}
\subsubsection*{This section is not numbered}
Nothing to see here... yet!
\section{Text formatting}
In Section \ref{sect.myfirst} we looked at in line equations.
Good?\\ Keep these lines or exercises in a few
sections time will not make sense!
\section{Tables and Arrays}
Here is a table.
\begin{center}
\begin{tabular}{r|cl} %pipe adds a separator line for tabular
%& used to move the headers around the vertical seperator
%double \\ tells to add a carriage return
number ($x$) & $x^2$ & reciprocal ($1/x$)\\
\hline %adds horizontal
1 & 1 & 1\\
2 & 4 & 0.5

\end{tabular}
\end{center}
There was a table.

Here is a table.\\
It’s a very happy table.
\begin{table}[h]
\begin{center}
\begin{tabular}{r|cl}
number $x$ & square $x^2$ & reciprocal $1/x$\\
\hline
1 & 1 & 1\\
\hline
2 & 4 & 0.5\\
\end{tabular}
\end{center}
\caption{A few numbers and powers.}
\label{tab.myfirst}
\end{table}
There was  table \ref{tab.myfirst}.

Below is the exercise 5 table:\\
\newline
\begin{table}[h]
\begin{tabular}{c|c|c}
\multicolumn{3}{c}{Table 1: Example of table with multi column and multi row} \\
\hline
\multirow{2}{*}{} Method & Sample Size & Prob to select $\boldsymbol{M}$ variables \\
& & \textit{M=1 M=2 M=3 M=4} \\
\hline
& Scenario 1: Two Correct Variables \\
\hline
\multirow{2}{*}{} The Best Method & 100 & 50\% 50\% 0\% 0\%  \\
& 200 & 75\% 25\% 0\% 0\% \\
\hline
\multirow{2}{*}{} The Worst Method & 200 & 0\% 0\% 50\% 50\% \\
& 200 & 0\% 0\% 25\% 75\% \\
\hline

\end{tabular}
\end{table}